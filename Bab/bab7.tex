%versi 2 (8-10-2016)
\chapter{Kesimpulan dan Saran}
\label{chap:kesimpulan}
Bab ini membahas kesimpulan dari pengerjaan tugas akhir ini dan saran untuk kedepannya.

\section{Kesimpulan}
\label{sec:kesimpulan} 
Selama pengambilan mata kuliah tugas akhir di DNArtworks, penulis mendapatkan kesimpulan bahwa pembangunan sistem pemesanan tiket kereta api membutuhkan banyak persiapan dalam pembangunannya. Contoh persiapan yang harus dilakukan adalah mencari dan membandingkan berbagai situs pemesanan tiket kereta lainnya dan menguji juga memahami \textit{API} secara menyeluruh dari awal. Mengetahui struktur \textit{framework} yang digunakan sangat membantu pembangunan dengan membuat kode yang lebih rapi dan terstruktur. Selain itu, mempelajari \textit{framework} CodeIgniter dan \textit{design pattern} MVC-nya sangat mempermudah pembangunan karena mencari atau memanfaatkan kembali kode yang sudah ada menjadi jauh lebih mudah.

\section{Saran}
\label{sec:saran}
Sistem masih memiliki banyak ruang untuk lebih disempurnakan. Pemanfaatan basis data yang lebih efektif untuk mengurangi waktu yang habis dari melakukan request yang sama berulang-ulang pada \textit{API}. Penggunaan \textit{loading screen} seperti pada halaman penerbangan untuk memberi informasi pada pengguna kalau permintaan sedang diproses saat waktu \textit{loading} cukup lama.
